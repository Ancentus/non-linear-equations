\documentclass{article}
\usepackage{amsmath}
\usepackage{listings}
\usepackage{hyperref}
\hypersetup{
    colorlinks=true,
    linkcolor=blue,      
    urlcolor=cyan
    }

\title{Solving Nonlinear Equations using Newton-Raphson and Bisection Methods}
\author{Ancentus Makau \\ Jomo Kenyatta University of Agriculture and Technology \\ Registration Number: SCT211-0469/2021}
\date{}

\begin{document}

\maketitle

GITHUB LINK: \url{https://github.com/Ancentus/non-linear-equations}

\section{Introduction}

We will solve the following nonlinear equation:

\begin{equation}
    f(x) = x^5 + x^3 - 2x - 5 = 0
\end{equation}

We will use the Newton-Raphson method and the bisection method to find the root of the equation.

\section{Newton-Raphson Method}

The Newton-Raphson method is an iterative method for finding the root of a function. The method starts with an initial guess $x_0$ and uses the derivative of the function $f'(x)$ to improve the guess. The iteration formula is:

\begin{equation}
    x_{n+1} = x_n - \frac{f(x_n)}{f'(x_n)}
\end{equation}

To use the Newton-Raphson method to solve Equation (1), we need to find the derivative of the function $f(x)$:

\begin{equation}
    f'(x) = 5x^4 + 3x^2 - 2
\end{equation}

We will start with an initial guess of $x_0 = 2.0$.
The Python program to implement the Newton-Raphson method is shown below:

\begin{lstlisting}
def newton_raphson(x0, tol=1e-6, max_iterations=100):
    """
    This function implements the Newton-Raphson method to solve the nonlinear equation f(x) = 0.
    x0: initial guess
    tol: tolerance for convergence
    max_iterations: maximum number of iterations to perform
    """
    x = x0
    for i in range(max_iterations):
        fx = f(x)
        dfx = 5*x**4 + 3*x**2 - 2
        if abs(fx) < tol:
            return x
        x = x - fx / dfx
    return None

x0 = 2.0
start_time = time.time()
root = newton_raphson(x0)
elapsed_time = time.time() - start_time
if root is not None:
    print(f"The root found by Newton-Raphson is: {root:.6f}")
    print(f"Time taken by Newton-Raphson: {elapsed_time:.6f} seconds")
else:
    print("Newton-Raphson failed to converge.")

\end{lstlisting}
Running the program gives the following output:

% output

The root found by Newton-Raphson is: 1.385768

Time taken by Newton-Raphson: 0.000028 seconds

\section{Bisection Method}

The bisection method is another iterative method for finding the root of a function. The method works by repeatedly dividing an interval in half and selecting the subinterval that contains the root. The iteration formula is:

\begin{equation}
    x_{n+1} = \frac{a_n + b_n}{2}
\end{equation}

where $a_n$ and $b_n$ are the endpoints of the subinterval at iteration $n$. To use the bisection method to solve Equation (1), we need to find two initial points $a_0$ and $b_0$ such that $f(a_0)$ and $f(b_0)$ have opposite signs.


The Python program to implement the bisection method is shown below:

\begin{lstlisting}
def bisection(a, b, tol=1e-6, max_iterations=100):
    """
    This function implements the bisection method to solve the nonlinear equation f(x) = 0.
    a: left endpoint of the initial interval
    b: right endpoint of the initial interval
    tol: tolerance for convergence
    max_iterations: maximum number of iterations to perform
    """
    fa = f(a)
    fb = f(b)
    if fa * fb > 0:
        return None
    for i in range(max_iterations):
        c = (a + b) / 2
        fc = f(c)
        if abs(fc) < tol:
            return c
        if fa * fc < 0:
            b = c
            fb = fc
        else:
            a = c
            fa = fc
    return None

a = 0.0
b = 3.0
start_time = time.time()
root = bisection(a, b)
elapsed_time = time.time() - start_time
if root is not None:
    print(f"The root found by bisection is: {root:.6f}")
    print(f"Time taken by bisection: {elapsed_time:.6f} seconds")
else:
    print("Bisection failed to converge.")

\end{lstlisting}

Running the program gives the following output:

% output

The root found by bisection is: 1.385768

Time taken by bisection: 0.000022 seconds

\section{Conclusion}

We have solved the nonlinear equation $f(x) = x^5 + x^3 - 2x - 5 = 0$ using the Newton-Raphson method and the Bisection method. Both methods give the same answer. The bisection method is slightly faster than the Newton-Raphson method for this equation.
\end{document}
